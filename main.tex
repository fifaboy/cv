\documentclass[11pt,a4paper]{moderncv}       

\moderncvstyle{casual}                             % style options are 'casual' (default), 'classic', 'oldstyle' and 'banking'
\moderncvcolor{blue}     
\usepackage[utf8]{inputenc}
\usepackage{fontawesome}
%\usepackage{fontspec}
\usepackage{tabularx}
\usepackage{ragged2e}

\usepackage[scale=0.8]{geometry}
\usepackage{graphicx}
\usepackage{multicol}
\usepackage{import}

\name{Masum}{Billal}
\photo[64pt][0.4pt]{pp1}                       % optional, remove / comment the line if not wanted; '64pt' is the height the picture must be resized 
\begin{document}

\makecvtitle
\begin{flushright}
	{\normalfont\normalsize Github: \texttt{https://github.com/fifaboy}}\\
	{\normalfont \normalsize Mobile: +8801779788023}\\
	{\normalfont\normalsize	Email: \texttt{billalmasum93@gmail.com}  } 
\end{flushright}

\section{Education}
\cventry{}{BSc}{CSE,Department of Computer Science and Engineering,University of Dhaka}{Bangladesh}{}{}  

\section{Work Experience}
\cventry{$2019$}{Senior Data Scientist, August $4$ to present}{Shohoz}{Dhaka}{}{At Shohoz, I am involved in design, planning and agile development of following: data pipeline in Azure, user profiling and clustering, data mining on a particular service, fraud analysis, priority dispatching and exploratory data analysis e.g. using \texttt{PySpark}, etc. Some of the services are already part of core Shohoz ecosystem e.g. periodical user profiling for quest planning. I have also worked on AWS based data pipeline using Kafka as message queue service.}
\cventry{$2019$}{Machine learning engineer and senior programmer, October $2018$ to May $2019$}{Auleek}{Dhaka}{}{During my time at Auleek, I used deep learning in tensorflow to extract features from floorplan images, worked on mesh generation and manipulation e.g. union/intersection, object placement in a given floorplan, etc.}
\cventry{$2017$}{Software engineer, R\&D, March, $2017$ to September, $2018$}{Reve systems}{Dhaka}{}{At Reve, I worked on variety of projects. I was part of a three man team responsible for a govt. project which can be found here: \url{http://evet.gov.bd/}. I also worked on a project called \textit{IP log} where my job was to receive and process all incoming and outgoing packets from a network interface. Multithreading was necessary to prevent packet loss.}
\cventry{$2016$}{Data analyst and software developer}{Remote Job}{Threat Equation PTE Ltd.}{Singapore}{My role was to predict vulnerability from server logs after training them on previously fed data. Then I had to present the predictions and related information in the dashboard for visualization.}

\section{Pet Projects}
\cventry{Seeding in \texttt{k-means} clustering}{Unsupervised learning}{Research}{Python}{I tried to improve the \texttt{k-means} clustering algorithm in this project}{\url{https://github.com/proafxin/clustering}}
\cventry{Localizer}{Object detection}{Personal project}{Python}{This project is a deep learning based localization application. It can be trained using given images and detect the objects in test images}{\url{https://github.com/proafxin/Localizer}}
\cventry{Improving collaborative filtering}{Machine learning}{Research on recommender Systems}{C++, Python}{In this project, I developed a recommender systems with F-1 score of almost 75\%. This was part of my BSc. thesis}{}
\cventry{Media recommender}{Machine learning based web application}{Personal project}{Django}{This is a web application that provides personalized feed based on user preferences and past history}{\url{https://github.com/proafxin/media-recommendation}}
\cventry{Online library}{Web application}{Personal project}{Dotnet core}{This is a dotnet core MVC based web application for online library.}{\url{https://github.com/proafxin/library}}

\section{Achievements and voluntary experience}
%\subsection{Vocational}
\cventry{$2013$}{Inter-University Programming Contest}{North South University, Bangladesh}{}{Rank $9$}{}
\cventry{$2012$}{Inter-University Programming Contest}{IUT, Bangladesh}{Rank $13$}{}{}
\cventry{$2013$}{ICPC}{Dhaka Regional, Bangladesh}{}{Rank $17$}{}{}
\cventry{$2012$}{ICPC}{Dhaka Regional, Bangladesh}{}{Rank $18$}{}{}
\cventry{$2010$}{Medal in math olympiad}{Bangladesh Mathematical Olympiad}{National, Regional}{}{}
\cventry{$2009$}{Medal in math olympiad}{Bangladesh Mathematical Olympiad}{Regional}{}{}
\cventry{$2011-2015$}{Trainer at math camps}{Bangladesh Mathematical Olympiad}{National, Extension and IMO math camps}{}{}
\cventry{$2019$}{Mathematical programming}{Global Rank $132$ on Hackerrank mathematics section}{}{}{}

\section{Publications}
\cventry{$2020$}{On Special Integer Sequences: Divisible, Lucas, Lehmer and Recurrent Sequence}{Masum Billal}{Currently under discussion with Springer for publication}{}{}
\cventry{$2019$}{Topics in Number Theory: An Olympiad Oriented Approach}{Masum Billal, Amir Hossein}{Amazon}{ISBN-10: 1719920311}{}
\cventry{$2015$}{Similarity Aggregation for Collaborative Filtering}{AIST Conference}{Sheikh Muhammad Sarwar, Masum Billal, Mahmudul Hasan, Dimitry I. Ignatov}{}{\texttt{http://link.springer.com/chapter/10.1007/978-3-319-26123-2\_23}}
\cventry{$2014$}{A Nice Theorem in Multiplicative Functions}{Eureka, The Cambridge University Mathematical Society}{Issue $63$}{Masum Billal}{}
\cventry{$2012$}{Exponent GCD Lemma}{Mathematical Reflections}{Issue $6$}{Masum Billal}{}
%\cventry{$2015$}{General LTE Sequence}{Arxiv E-print}{Masum Billal}{My paper on a generalization of Lifting the Exponent property in sequence}{\texttt{https://arxiv.org/abs/1509.03288}}
%\cventry{$2015$}{Remarks On General Fibonacci Numbers}{Arxiv E-print}{Masum Billal}{My results on Fibonacci number}{\texttt{https://arxiv.org/abs/1502.06869}}
%\cventry{$2015$}{Combinatorial Geometry In Olympiad}{Accepted for Mathematical Reflections in $2015$}{Masum Billal, Nur Muhammad Shafiullah}{}{}
%\cventry{$2016$}{A Nice Lemma In Congruence: Thue’s Lemma}{Under review by Crux Mathematicorum, Canadian Mathematical Society}{Masum Billal}{}{}
\cventry{$2016$}{$10$th Bangladesh Mathematical Olympiad: Selected Problems and Solutions}{Bangladesh Math Booklet, $2015$}{}{}{\texttt{https://karushib.wordpress.com/2016/05/29/bangladesh-booklet-of-math-problems-2015/}}
%\cventry{$2015$}{Number Theory Problems From IMO}{Art of Problem Solving}{}{A collection of International Mathematical Olympiad problems}{\texttt{http://www.artofproblemsolving.com/community/c6h1161504p5531708}}
%\cventry{$2015$}{Collection of Number Theory Problems: APMO}{Art of Problem Solving}{Masum Billal}{A collection of Asian Pacific Mathematical Olympiad Number Theory problems}{\texttt{http://www.artofproblemsolving.com/community/q3h1160887p5526295}}
%\cventry{$2015$}{A Sledgehammer in Olympiad Number Theory: Zsigmondy’s Theorem}{}{}{Masum Billal}{}
%\cventry{$2015$}{Combinatory Problems}{}{}{Masum Billal}{A combinatorics tutorial in Bangla.}


\section{Interests and Hobbies}
\cvitem{Research}{Optimizing algorithms or finding new ones. Studying on topics I am interested in for interesting results. Mathematics, Algorithms, Cryptography, Machine Learning etc.}
\cvitem{Programming}{Creating and solving problems. Proposed some problems for Hackerrank and some inter university contests.}
\cvitem{Mathematics}{Finding new results and creating or solving new problems.}
\cvitem{Gaming}{Playing Dota $2$, Assassin Creed Origins etc.}

\section{Others}
\cvlistitem{LinkedIn: \texttt{https://www.linkedin.com/in/billalmasum93}}
\cvlistitem{Blog: \texttt{https://karushib.wordpress.com/}}


\begin{thebibliography}{99}
	\bibitem{farhad} Farhad Alam Bhuiyan, Data Science Lead, Shohoz, Bangladesh. Email: \texttt{farhad.alam@shohoz.com}
	\bibitem{majumdar} \textit{Mahbub Majumdar}, Professor and Chairperson, Department of Computer Science and Engineering, BRAC University. Email: \texttt{majumdar@bracu.ac.bd}
\end{thebibliography}
\nocite{*}
\end{document}


%% end of file `template.tex'.
